\documentclass{ProcISPRAS}

% Page geometry. Nearly A5, but not exactly
\usepackage[papersize={14.86cm,21cm},
            left=1.5cm, % 1.4cm
            right=1cm, % 1.5cm
            top=0.8cm, % 0.5cm
            bottom=1cm, % 1.5cm
            includehead,
            headheight=8pt,
            heightrounded,
            headsep=6pt, % 0.4cm
            includefoot,
            footskip=16pt,
]{geometry}

\addbibresource{bibliography.bib}

\usepackage{url}
\usepackage{tikz}
\usepackage{booktabs}
\usepackage{listings}
\usetikzlibrary{positioning}
\usepackage{algorithmicx}
\usepackage{algpseudocode}

\volhead{1}
\issuehead{2}
\pageshead{3--4}
\yearhead{2018}

\setcounter{page}{1} % Set first page number

\date{June 2018}

\newauthor
\inst{1}
\authorname{Ivanov I.\,I.}{Иванов~И.\,И.}
\orcid{0000-0001-7705-7214}
\email{ivanov@ispras.ru}

\newauthor
\inst{1,2}
\authorname{Petrov P.\,P.}{Петров~П.\,П.}
% ORCID is optional
%\orcid{0000-0040-0005-7012}
\email{petrov@ispras.ru}


\affil[1]{
Ivannikov Institute for System Programming of the RAS,\\
25, Alexander Solzhenitsyn Str., Moscow, 109004, Russia}{
Институт системного программирования им. В.П. Иванникова РАН,\\
109004, Россия, г. Москва, ул. А. Солженицына, д. 25}

\affil[2]{
Lomonosov Moscow State University,\\
GSP-1, Leninskie Gory, Moscow, 119991, Russian Federation}{
Московский государственный университет имени М.В. Ломоносова,\\
119991, Россия, Москва, Ленинские горы, д. 1.}

\title{Paper title in English}{Заголовок статьи для Трудов ИСП РАН}

\doi{10.15514/ISPRAS-2018-1(2)-33}
\keywords{keywords in English}{список ключевых слов, разделенных точкой с запятой.}

% Acknowledgments are optional
%\acknowledgments{This work was supported by a grant from the Russian Science Foundation 18-11-00100}{Данная работа выполнена при поддержке гранта РНФ 18-11-00100}

\abstract{% Abstract in English
Следует понимать, что англоязычная аннотация~--- это единственный источник
информации о статье для англоязычных читателей (и реферативных баз данных); она
не обязана быть дословным переводом аннотации на русском языке. Объем аннотации
должен составлять 200-300 слов. Аннотации не должны содержать ссылок на
литературу, аббревиатуры (если возможно), лишних вводных фраз и сложных
грамматических конструкций. Аннотация должна описывать основные цели работы,
объяснять, как было проведено исследование (но без лишних деталей), суммировать
наиболее важные результаты и их значимость. Аннотация должна быть
самодостаточной: в ней не должно быть фраз типа ``полученные результаты
описываются в последнем разделе статьи''~--- результаты должны кратко
описываться в самой аннотации. Приветствуются аннотации, структура которых
повторяет структуру статьи и включает в сжатой форме введение, цели и задачи,
методы, результаты и заключение.}{ % Русская аннотация
Различные компоненты Вашей статьи (заголовок статьи, авторы, заголовки частей)
определены в таблице стилей. Примеры Вы можете видеть в оформлении этого
шаблона. Красным цветом указан стиль оформления каждого компонента.}


\begin{document}

\makedoi

\maketitleen

\newpage

\maketitleru

\section{Введение}

Приведенный ниже материал поможет Вам подготовить текст статьи для Трудов
Института системного программирования РАН.

Во избежание ошибок при форматировании текста статьи настоятельно рекомендуется
использовать данный документ в качестве шаблона. Это позволит получать все
заданные параметры форматирования текста автоматически. В противном случае
необходимо самостоятельно обеспечить выполнение всех требований данного
документа (размер страницы, поля и отступы, шрифт, расстояние между колонками и
т.\,д.).

В Трудах ИСП РАН публикуются только оригинальные статьи, ранее не
публиковавшиеся в других изданиях. Объем публикуемых статей, как правило, не
должен превышать 20 страниц.

Публикуемые в Трудах ИСП РАН статьи состоят из следующих последовательно
расположенных элементов:
\begin{itemize}
  \item название статьи;
  \item инициалы и фамилии авторов;
  \item электронные адреса авторов;
  \item полное название и адрес организации (с обязательным указанием страны и
    города);
  \item аннотация;
  \item ключевые слова;
  \item текст статьи;
  \item список использованных публикаций (заголовок <<Список литературы>>).
\end{itemize}

В конце статьи должны приводиться название статьи, инициалы и фамилии авторов,
электронные адреса авторов, полные названия организаций и их адреса с указанием
города и страны, аннотация и ключевые слова, а также список <<References>> на
английском языке.

\section{Обзор форматирования}

В этом разделе раскрыты характеристики стилей, используемых в данном документе.

\subsection{Требования по шрифтам}

Общие требования к оформлению:
\begin{itemize}
  \item основной шрифт~--- Times New Roman;
  \item шрифт для заголовков~--- Arial;
  \item междустрочный интервал~--- одинарный;
  \item основной размер шрифта~--- 10 пт.;
  \item отступы абзаца отсутствуют, абзацы не разделяются пустой строкой;
  \item выравнивание <<по ширине>>.
\end{itemize}

Заголовок статьи (заголовок первого уровня) вводится полужирным шрифтом Arial
размера 16 пт. Задается выравнивание по центру.

Заголовки второго уровня (к ним относятся названия разделов статьи, первым из
которых должен быть раздел "Введение", а последним~--- "Заключение") вводятся
полужирным курсивным шрифтом Arial размера 11 пт. Заголовки разделов нумеруются
с точкой после номера раздела (1., 2. и т.д.) и не выравниваются. До заголовка
второго уровня помещается пустая строка.

Заголовки третьего и четвертого уровней (названия подразделов и пунктов статьи)
вводятся полужирным шрифтом Arial размера 11 и 10 пт. соответственно. Заголовки
подразделов и пунктов нумеруются без точки в конце номера (1.1, 1.1.2 и т.д.) и
не выравниваются. До заголовка третьего или четвертого уровня помещается пустая
строка.

Наличие в статье заголовков пятого и больших уровней не приветствуется. Если они
необходимы, решение об оформлении принимает автор. В любом случае такие
заголовки не должны нумероваться.

Инициалы и фамилии авторов, названия их организаций  и электронные адреса
авторов отделяются от названия статьи пустой строкой, вводятся шрифтом Times New
Roman размера 10 пт. «курсив» на последовательных строках и выравниваются по
центру.

Ключевые слова статьи (не более десяти) вводятся после аннотации. Заголовок
"Ключевые слова:" вводится полужирным шрифтом Times New Roman размера 9 пт., и
сразу вслед за ним вводится сам список ключевых слов (без использования
заглавных букв), разделенных точками с запятой.

\subsection{Оформление абзаца}

Выравнивание абзацев соответствует общепринятой практике~--- в основном тексте
<<по ширине>>, в заголовках~--- по центру (кроме заголовков подразделов).
Отступы и интервалы для различных стилей различны, но обратите внимание, что для
стиля «ispText\_main» должен использоваться одинарный межстрочный интервал.

\subsection{Поля}

Статью необходимо приготовить для бумаги формата А4 (21см х 29,7см). Размеры
полей:

\begin{tabular}{ll}
  Верхнее: 1 см & Нижнее: 1,5 см \\
  Левое: 1.4 см & Правое: 1.5 см \\
  Переплет: 0.  & \\
\end{tabular}

\subsection{Рисунки}

\begin{figure}[t]
  \centering
  \begin{tikzpicture}
  \node[draw,thick,rectangle,minimum width=3cm,minimum height=1.7cm,text width=3cm,align=center] (rect) {Прямоугольник\\Rectangle};
  \node[draw,thick,circle,right=1cm of rect,text width=2cm,align=center] (circ) {Круг\\Circle};
  \node[draw,thick,rectangle,right=1cm of circ,minimum width=2cm,minimum height=2cm,text width=2cm,align=center] (square) {Квадрат\\Square};
  \end{tikzpicture}
  \captionisp{Геометрические фигуры}{Geometric shapes}
  \label{fig:my_label}
\end{figure}

\begin{table}[t]
  \captionisp{Размер кода системы Posh}{Posh system code size}
  \begin{tabular}{cc}
  \toprule
  \textbf{Версия} & \textbf{Размер, MLOC} \\
  \midrule
  2.0             & 1.42 \\
  3.0             & 1.97 \\
  \bottomrule
  \end{tabular}
  \label{tbl:my_label}
\end{table}

Надписи на рисунке по возможности должны быть на русском и английском языках.
Подрисуночные подписи должны быть на двух языках и начинаться с текста вида
"Рис. 1" ("Fig. 1"), заканчиваться точкой и выравниваться по центру. При наличии
в статье таблиц надтабличная надпись на двух языках ставится над таблицей,
начинается с текста вида "Табл. 1" ("Table 1")  и не выравнивается. Ссылки на
рисунки и таблицы в тексте статьи должны иметь вид "рис. 1" и "табл. 1"
соответственно.

\begin{listing}[t]
\begin{lstlisting}[language=C,frame=none,basicstyle=\ttfamily]
#include <stdio.h>

int main(void)
{
    printf("Hello, world!\n");
    return 0;
}
\end{lstlisting}
\centering
\captionisp{Пример листинга программы}{Sample program listing}
\label{lst:my_label}
\end{listing}

\begin{algorithm}[t]
\captionisp{Алгоритм офисного работника}{Office worker algorithm}
\begin{algorithmic}
\Procedure{Office-Worker-Algorithm}{\null}
  \ForAll{$day\in\{mon, tue, wed, thu, fri\}$}\Comment{Рабочий день}
    \While{$day$ is not over}
      \State \Call{Do-Some-Work}{\null}
      \State \Call{Drink-Some-Coffee}{\null}
    \EndWhile
  \EndFor
\EndProcedure
\end{algorithmic}
\label{alg:my_label}
\end{algorithm}

\section{Список литературы}

Особое внимание следует уделять правильному оформлению списка используемой
литературы и соответствующих ссылок на источники внутри статьи. Список
литературы должен строиться из библиографических ссылок на используемые
публикации, список нумеруется в последовательности использования ссылки на
соответствующий источник в тексте статьи (т.\,е. первой в этом списке должна
содержаться библиографическая  ссылка, первой упоминаемая в тексте статьи). Для
ссылки на источник в тексте используется номер соответствующей библиографической
ссылки в квадратных скобках. При этом самоцитирование не должно превышать 30\%.

Библиографические ссылки должны включать следующие данные:
\begin{itemize}
  \item Для книги~--- фамилии и инициалы всех авторов; полное название книги;
    наименование издательства и город, в котором оно находится; год издания;
    количество страниц книги;
  \item Для статей~--- фамилии и инициалы всех авторов; полное название статьи;
    название журнала, газеты или сборника, в котором (которой) опубликована
    статья; год издания, идентификатор времени публикации (для газеты~--- номер
    выпуска или дата выхода, для журнала~--- год, том или номер выпуска, серия),
    номера страниц, занятых статьей (начальная и конечная). Если для публикации
    известен идентификатор DOI (Digital Object Identifier), его следует
    поместить в конце ссылки (например, doi: 10.1134/S03617688060400). Для всех
    статей в «Programming and Computer Software», а также для статей из Труды
    Института Системного программирования РАН, начиная с 22 выпуска, такие
    идентификаторы есть. Если статья есть в реферативной базе данных Scopus, то
    необходимо брать данные из самой статьи.
  \item Для стандартов~--- название стандарта, номер стандарта, место и год
    издания, страницы;
  \item Для патентных документов~--- название изобретения; номер патента;
    страна, номер и дата заявки на изобретение, дата опубликования патента;
    номер бюллетеня изобретений, страницы;
  \item Для депонированных научных работ~--- фамилии и инициалы всех авторов;
    полное название работы; название депонирующего информационного центра; номер
    и дата депонирования; количество страниц работы;
  \item Для диссертаций~--- фамилии и инициалы автора, полное название
    диссертации; на соискание какой ученой степени представлена диссертация;
    место и год защиты диссертации; количество страниц диссертации;
  \item Для электронных ресурсов в Интернет~--- фамилии и инициалы всех авторов
    (если они известны), полное название материала, полный электронный адрес
    (например, \url{http://www.ispras.ru/ru/proceedings/authors.php}), дата
    публикации или создания, дата обращения (если невозможно установить дату
    публикации или создания).
\end{itemize}

Названия книг, статей, иных материалов и документов, опубликованных на
иностранном языке, а также фамилии их авторов должны быть приведены в
оригинальной транскрипции. В список используемой литературы не должны включаться
неопубликованные материалы или материалы, не находящиеся в общественном доступе.

Примеры оформления списка литературы~\cite{ermakov15, chervyakov17,
proskuryakova17, burgess03, naveh07, dijkstra76, un210-kit, akushsky68}:

\begin{otherlanguage}{english}
\printbibliography
\end{otherlanguage}

{\vskip 12pt\normalfont\sffamily\bfseries\large Информация об авторах / Information about authors}
\setlength{\parskip}{6pt}

Иван Иванович ИВАНОВ~--- доктор технических наук, профессор, заведующий отделом
прикладной математики и информатики Института системного программирования с 2004
года. Сфера научных интересов: алгебраические структуры в полях Галуа,
модулярная арифметика, нейрокомпьютерные технологии, цифровая обработка
сигналов, криптографические методы защиты информации.

Ivan Ivanovich IVANOV~--- Doctor of Technical Sciences, Professor, Head of the
Department of Applied Mathematics and Computer Science of the Institute for
System Programming of the RAS since 2004. Research interests: algebraic
structures in the Galois fields, modular arithmetic, neurocomputer technologies,
digital signal processing, cryptographic methods for protecting information.

Петр Петрович ПЕТРОВ является специалистом кафедры Системного программирования
Московского государственного университета имени М.В. Ломоносова. Его научные
интересы включают распознавание образов, системы остаточных классов.

Petr Petrovich PETROV is a specialist of the Department of system programming of
CMC of Lomonosov Moscow State University. His research interests include pattern
recognition, residual class systems.

\end{document}
